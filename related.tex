\section{Related Work}
\label{sec:related}

\nparagraph{Measuring path changes.}  Several previous studies have
introduced new path measurement techniques~\cite{as-relationships,
shaikh:nsdi04, hubble, iplane, planetseer, oliveira09exploration,
revtr}, \eg to discover inter-AS links, quantify BGP path exploration,
or infer an AS's path preferences.  Our system combines multiple sources
of data to collect Internet path measurements, including public
BGP feeds~\cite{routeviews} and traceroute measurements collected from
PlanetLab.  
\begin{comment}
Less widespread is our use of Reverse
Traceroute~\cite{revtr} to collect path measurements from arbitrary ASes
back to monitored prefixes.  Reverse Traceroute sends IP Record Route
probes to a target router from distributed vantage points spoofing as
the source, so that the source receives a sequence of routers on the
path back from the target router to itself.  Even though Reverse
Traceroute does not work on paths that filter IP Record Route probes or
spoofed packets, it helps us cover ASes invisible from BGP feeds or
traceroutes. 
\end{comment}
Previous work has used BGP poisoning to quantify the use
of default routes in the Internet~\cite{optometry}.  We use BGP
poisoning to perform topology discovery in a way equivalent to that of
Colitti \emph{et al.}~\cite{lorenzo-thesis}.  However, ours is the first
work we are aware of that uses BGP poisoning to obtain ground truth for
evaluating root cause analysis in the Internet.  

% We also use poisoning to aid in the inference of path preference in
% the Internet, similar to how Oliveira \emph{et al.}~inferred path
% preference from the sequence of updates during the BGP convergence
% process~\cite{oliveira09exploration}.  

% In the future, we plan to other work on inferring path preference into
% our system~\cite{oliveira09exploration, as-relationships} and further
% explore the use of BGP poisoning for path preference inference.  We
% note that our system does not, but could, use data collected from
% intra-AS routing protocols such as OSPF and
% IS-IS~\cite{shaikh:nsdi04}.  Intra-AS protocol information would be
% useful for more fine-grained root cause identification.  One challenge
% is that ASes are unwilling to share routing information; existing
% approaches that use intra-AS routing information are limited to a
% single network~\cite{ipapproach, netdiagnoser}, but our goal is to
% identify the root cause of path changes in the wide-area Internet.  

\nparagraph{Impact of path changes.} Several previous studies highlight
the impact of route
changes~\cite{labovitz00,feamster,beacons,feamster07,ratul}, \eg causing
increased delays and transient packet loss due to convergence, in
addition to outages due to misconfiguration. In this work, we identify
the root cause of such pathological changes, which can assist in
debugging and correcting the problem. Previous work also studied how
paths propagate in the Internet~\cite{griffin-wilfgong-sheperd,wedgies}.
Our work uses a basic routing model to understand how interdomain path changes
propagate, building on previous work that models the
BGP decision process~\cite{gao}.  

\nparagraph{Root cause analysis.} As described in
\S\ref{sec:challenges}, the work by Feldmann \emph{et
al.}~\cite{feldman} and Caesar \emph{et al.}~\cite{caesar} are most
closely related to ours. We demonstrated that the \feldmann assumption
prevents their approaches from achieving both accuracy and precision.
Several other papers have looked at the problem of root cause analysis
from different perspectives.  Wu et al.~\cite{needle} propose a system
to identify high-impact network events affecting an AS, using routing
changes observed at an AS's border routers and correlating it with
traffic changes.  The system helps operators identify important events,
but is limited to one network and cannot identify the cause of events.
In particular, if an event is caused by a distant AS, their system can
only indicate the border routers involved. Pei et al.~\cite{BGP-RCN}
propose a modification to BGP that includes information about the
networks responsible for a path change. Our work demonstrates that one
can identify root causes without requiring protocol modification.
LIFEGUARD~\cite{lifeguard} and NetDiagnoser~\cite{netdiagnoser}
identified a set of routers responsible for packet-forwarding outages,
but did not focus on which network's path change (if any) caused the
outage.  




%\begin{itemize}
%\item \cite{feldman} uses AS level information; is passive; assumes
%  problems either on the old path or the new path.
%
%\item \cite{caesar, caesar-root-cause} similar to \cite{feldman}.  Correlation over time
%  and across destinations.
%
%\item \cite{planetseer} is a passive + active monitoring technique for
%  path changes, but does not provide root cause information.
%
%\item \cite{hubble} focuses on reachability problems, provides
%  rudimentary root cause information.
%
%\item \cite{iplane} monitors routing data over time but does not
%  analyze path changes across time.
%
%\item \cite{teixeira} discusses a measurement framework that could
%  help with root-cause information.
%
%\item \cite{BGP-RCN} augments BGP with root cause information.
%
%\item Wu et al.~\cite{needle} propose a system to identify high-impact
%network events affecting an AS.  They aggregate path changes observed
%from an AS's border routers into network events using characteristics of
%the observed changes (e.g., whether the router changed to a more or less
%preferred route).  They also use per-prefix traffic measurement at
%external links to compute how much traffic changed routes due to the
%event and estimate its impact.  The system helps operators identify
%important events, but is limited to one network and cannot identify the
%cause of events.  In particular, if an event is caused by a distant AS,
%their system can only indicate the border routers involved.
%
%\item Other refs: \cite{netdiagnoser}, \cite{feamster},
%  \cite{labovitz00}, \cite{ratul}, \cite{beacons}, \cite{darkspaces},
%  \cite{shaikh:nsdi04}, \cite{ipapproach,kompella-infocom},
%  \cite{feamster07}. 
%
%\end{itemize}
