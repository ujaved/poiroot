\section{Background}
\label{sec:bg}
In this paper, we focus on the problem of identifying the AS 
responsible when an interdomain path change impacts 
network connectivity or performance. Previous work focuses 
on complementary problems such as detecting path 
changes~\cite{feamster,beacons}, identifying long-lasting network and performance problems~\cite{hubble,planetseer,netdiagnoser,news} 
and finding alternative paths that avoid these problems~\cite{ron,lifeguard}. 
Instead of simply avoiding a problem when it occurs, our goal is to 
provide salient information about its root cause so as to reduce time to repair.

\subsection{Previous Work}

A number of assumptions underly this simple algorithm, and the authors 
present heuristics to cope with some cases where the assumptions are 
violated. With these heuristics in place, they were able to isolate the origin 
of a BGP path change to a single AS or interdomain link for 96\% of 
the cases they study. While these results may suggest that we can close 
the book on root cause analysis, the following section discusses 
several limitations that motivate us to reopen it.
  
\subsection{Limitations}

The algorithm and methodology proposed by Feldmann et al. 
can in many cases correctly identify a narrow region of the network 
that could have caused a BGP path change. However, several 
limitations significantly limit the accuracy and coverage of their 
approach. We now discuss these limitations and how we address 
them in our work.

\textbf{Simplifying assumptions.} Two key assumptions in 
previous work is that all Internet paths are valley free and 
that shortest paths are always preferred over longer ones. 
While it is well known that these assumptions are violated in 
practice it is difficult to determine \emph{when} and \emph{where} 
they are violated. In this work, we announce paths that allow 
us to determine the next-hop routing preferences for an AS toward 
a prefix we control. We then incorporate this empirical information 
in our root cause analysis algorithm. \tbd{Add results showing 
this is the case}

\textbf{Induced path changes.} \label{sec:induced}Another key assumption in 
previous work is that the origin of a path change appears 
either in the old path or the path. This assumption is false 
in general, leading to so-called induced path changes 
where the origin of path change is in neither the old path 
nor the new path. \tbd{Add a simple example.} Previous work ignores such cases; in this work 
we not only account for such cases when performing root 
cause analysis, we also perform the additional measurements 
required to identify the root cause when these cases occur. 
\ekb{I don't think this is a fair characterization of previous work. Feldmann's
paper specifically points out induced path changes in Figure 3(d) and has a
section labeled ``Caution regarding induced updates'' that starts ``Figure 3(d) shows the dan-
ger of assuming that the origin of all instabilities is in either on the new or
on the old path.'' Also, we have Feldmann spelled incorrectly in many places -- it has two `n's.}

\textbf{Lack of ground truth.} By their very nature, interdomain 
path changes cross multiple administrative domains and thus 
present a challenge when attempting to determine ground-truth 
information. In particular, ISPs generally consider routing decisions to 
be proprietary and confidential information, so even when an 
algorithm isolates a path change it is difficult to confirm its accuracy. 
In this work, we obtain ground truth because we can induce path 
changes at links in the real Internet and located outside our AS.  

\textbf{Lack of control.} Previous work takes a passive approach to 
observing path changes in the real Internet. As a result, evaluations 
are strictly limited to path changes that happened to occur during the 
observation period and in regions of the network made visible to 
researchers. In this work, we make announcements that 
allow us to control when and where paths change. This allows us to 
more generally understand the nature of path changes in response 
to link unavailability toward prefixes we control thus improving our 
ability to identify the root cause of path changes that naturally occur 
toward those prefixes.

\textbf{Limited visibility.} Identifying the root cause of 
path change requires information about paths before and after a change; 
thus we would like a continuous feed of routing information from as 
many vantage points as possible. Until recently, researchers 
could use only a limited set of available BGP feeds (e.g., RouteViews and 
RIS) or traceroute data collected from a set of distributed vantage 
points that support continuous measurement (e.g., PlanetLab). In 
this work, we augment these datasets with path information gathered 
from Reverse Traceroute. As we show in Sec.~\ref{}, this additional 
information allows us to identify the root cause even in the case 
of an induced path change.


%\tbd{DRC: This is redundant-ish with the next subsection but I wanted 
%to try to reuse at least some this text here.}
%Feldmann et al.~\cite{feldman} used simulation and passive measurement to
%isolate the locations of origins for routing instability. Our
%hypothesis is that we can improve on this work by addressing the
%following limitations: i) developing a new theory of path-change 
%origins to include a more complete set of candidate ASes. ii) Incorporating not only passive BGP feeds but
%also active measurements will give us a more accurate view of paths
%actually being used; iii) correlation of measurements across large
%numbers of vantage points made available through revtr will help us
%better isolate route change origins, even so-called induced updates;
%iv) using poisoning, we can conduct controlled experiments to better
%understand how and why paths change in response to link
%unavailability, potentially revealing previously opaque routing
%policies; v) existing BGP simulations suffer from a lack of
%information about route preferences, and the results of our controlled
%experiments can potentially provide insight on how to better capture
%this information; vi) path changes ``in the wild'' are subject to noise
%and misinformation; using our controlled experiments we can further
%refine path change identification and analysis heuristics.




